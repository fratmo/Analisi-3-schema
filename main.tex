\documentclass[11pt,a4paper]{article}

% ====== Pacchetti essenziali ======
\usepackage[italian]{babel}
\usepackage[T1]{fontenc}
\usepackage[utf8]{inputenc}
\usepackage{lmodern} % font leggibile
\usepackage{amsmath,amssymb,amsthm,mathtools}
\usepackage{physics} % \dv,\pdv,\abs,\norm
\usepackage{siunitx}
\usepackage{enumerate}
\usepackage{bm}
\usepackage{geometry}
\usepackage{tikz}
\usepackage{pgfplots}
\usepackage{titlesec}
\usepackage{tcolorbox}
\usepackage{hyperref}
\hypersetup{colorlinks=true, linkcolor=blue, urlcolor=blue, citecolor=blue}
\geometry{margin=2.2cm}
\pgfplotsset{compat=1.18}
\tikzset{>=latex}

% ====== Macro utili ======
\newcommand{\C}{\mathbb{C}}
\newcommand{\R}{\mathbb{R}}
\newcommand{\e}{\mathrm{e}}
\newcommand{\ii}{\mathrm{i}}
\newcommand{\Arg}{\operatorname{Arg}}
\newcommand{\sgn}{\operatorname{sgn}}
\DeclareMathOperator{\Res}{Res}

% ====== Ambienti stile esame ======
\newtcolorbox{defbox}{colback=blue!3!white,colframe=blue!50!black,left=1mm,right=1mm}
\newtcolorbox{ideabox}{colback=yellow!8!white,colframe=orange!70!black,left=1mm,right=1mm}
\newtcolorbox{exambox}{colback=green!6!white,colframe=green!50!black,left=1mm,right=1mm}
\newtcolorbox{alertbox}{colback=red!3!white,colframe=red!60!black,left=1mm,right=1mm}

\newtheorem{theorem}{Teorema}
\newtheorem{lemma}{Lemma}
\newtheorem{proposition}{Proposizione}
\theoremstyle{definition}
\newtheorem{example}{Esempio}
\newtheorem{remark}{Osservazione}

% Titoli compatti
\titleformat{\section}{\Large\bfseries}{\thesection}{0.6em}{}
\titleformat{\subsection}{\large\bfseries}{\thesubsection}{0.5em}{}
\titleformat{\subsubsection}{\bfseries}{\thesubsubsection}{0.5em}{}

% ====== FRONTESPIZIO ======
\title{\vspace{-1cm}\textbf{ANALISI 3 – Schemi Completi}\\
\large Numeri complessi \;|\; Serie di Fourier \;|\; Funzioni olomorfe}
\author{Preparato da: \textsc{Antonio Cioffi}}
\date{}

\begin{document}
\maketitle
\vspace{-0.8em}
\begin{ideabox}
\textbf{Struttura dell'esame (3 esercizi, 10 pt ciascuno):}
\begin{itemize}
  \item \textbf{Numeri complessi} (90\%: radici n-esime; a volte equazioni che si riducono a radici).
  \item \textbf{Serie di Fourier}: costruzione dei coefficienti, serie e studio di convergenza (puntuale e uniforme).
  \item \textbf{Funzioni di variabile complessa}: dominio, olomorfia (Cauchy–Riemann) $\Rightarrow$ analiticità.
\end{itemize}
\textbf{Regola d'oro per il voto pieno:} mostra \emph{tutti} i passaggi con ordine, dichiara le ipotesi e le conclusioni con frase finale esplicita.
\end{ideabox}

\tableofcontents

% =========================================================
\section{Numeri complessi: forme, potenze, radici, equazioni}
% =========================================================

\subsection{Forme equivalente di un numero complesso}
\begin{defbox}
Sia $z=a+\ii b\in\C$. Modulo $\abs{z}=\rho=\sqrt{a^2+b^2}$, argomento $\arg z=\theta$ (non unico), argomento principale $\Arg z\in(-\pi,\pi]$.\\
Forme:
\[
z=a+\ii b=\rho(\cos\theta+\ii\sin\theta)=\rho\,\e^{\ii\theta}.
\]
\end{defbox}

\begin{ideabox}
\textbf{Conversione rapida} $a,b\to(\rho,\theta)$: $\rho=\sqrt{a^2+b^2}$,\; $\cos\theta=\frac{a}{\rho}$,\; $\sin\theta=\frac{b}{\rho}$.\;
Usa $\tan\theta=\frac{b}{a}$ \emph{solo} per il valore grezzo, poi \textbf{correggi il quadrante}.
\end{ideabox}

\subsection{Formula di Eulero e di De Moivre}
\begin{defbox}
\[
\e^{\ii\theta}=\cos\theta+\ii\sin\theta,\qquad
\bigl(\rho\e^{\ii\theta}\bigr)^{n}=\rho^{n}\e^{\ii n\theta}=\rho^{n}\bigl(\cos(n\theta)+\ii\sin(n\theta)\bigr).
\]
\end{defbox}

\subsection{Radici $n$-esime di un complesso}
\begin{theorem}[Radici n-esime]
Sia $z=\rho\,\e^{\ii\theta}\neq0$. Le radici $n$-esime sono
\[
w_k=\sqrt[n]{\rho}\;\e^{\ii\frac{\theta+2k\pi}{n}}
=\sqrt[n]{\rho}\Bigl(\cos\frac{\theta+2k\pi}{n}+\ii\sin\frac{\theta+2k\pi}{n}\Bigr),\quad k=0,1,\dots,n-1.
\]
Sono $n$ punti ai vertici di un poligono regolare su cerchio di raggio $\sqrt[n]{\rho}$, equispaziati di $\frac{2\pi}{n}$.
\end{theorem}

\begin{ideabox}
\textbf{Schema d’esame per radici}:
\begin{enumerate}[(1)]
\item Porta $z$ in forma trig./esponenziale ($\rho$, $\theta$ con \emph{quadrante giusto}).
\item Scrivi la formula generale $w_k$ e specifica $k=0,\dots,n-1$.
\item Se richiesto, \textbf{ridai} le soluzioni in forma $a+\ii b$ (arrotonda solo alla fine).
\item Disegna velocemente: cerchio di raggio $\sqrt[n]{\rho}$, marca gli $n$ angoli. Punti $\checkmark$.
\end{enumerate}
\end{ideabox}

\subsubsection*{Mini–figura pronta (radici quarte)}
\begin{center}
\begin{tikzpicture}[scale=1]
\draw[->] (-2.2,0)--(2.2,0) node[right]{$\Re$};
\draw[->] (0,-2.2)--(0,2.2) node[above]{$\Im$};
\draw (0,0) circle (1.6);
\foreach \k in {0,1,2,3}{
  \coordinate (P\k) at ({1.6*cos(45+90*\k)},{1.6*sin(45+90*\k)});
  \fill (P\k) circle (2pt);
}
\node at (1.1,1.2){$w_0$}; \node at (-1.2,1.1){$w_1$};
\node at (-1.1,-1.2){$w_2$}; \node at (1.2,-1.1){$w_3$};
\end{tikzpicture}
\end{center}

\subsection{Rationalizzazione coniugata (\emph{trucco salva-tempo})}
\begin{ideabox}
Per eliminare complessi al \textbf{denominatore}, moltiplica per il \emph{coniugato}:
\[
\frac{A+\ii B}{C+\ii D}
=\frac{(A+\ii B)(C-\ii D)}{C^2+D^2}=\frac{AC+BD}{C^2+D^2}+\ii\,\frac{BC-AD}{C^2+D^2}.
\]
Sempre porta il risultato in $a+\ii b$ prima di passare alla forma trig./exp.
\end{ideabox}

\subsection{Equazioni tipiche nel campo complesso}
\begin{exambox}
\textbf{Template 1 (si riduce a radici)}: risolvi $z^n=\rho\,\e^{\ii\theta}$ con formula $w_k$.\\[0.25em]
\textbf{Template 2} $z^m\overline{z}^{\,n}=c$ (\;con $c\in\C$):
scrivi $z=\rho\e^{\ii\theta}\Rightarrow \overline{z}=\rho\e^{-\ii\theta}$. Allora
$z^m\overline{z}^{\,n}=\rho^{m+n}\e^{\ii(m-n)\theta}=c$.
Confronta moduli e argomenti:
\[
\rho^{m+n}=\abs{c},\qquad (m-n)\theta\equiv \Arg c \pmod{2\pi}.
\]
Poi ricava $z$.
\end{exambox}

\begin{example}[Stile compito]
Risolvi $z^2\overline{z}=-16\overline{z}$ (caso tipico). \\
\emph{Sol.} Se $\overline{z}=0$ allora $z=0$, che soddisfa l'equazione ($0=-16\cdot0$). 
Se $\overline{z}\neq0$ si può dividere: $z^2=-16$. 
In forma exp: $-16=16\e^{\ii(\pi+2k\pi)}$. Radici quadrate:
\[
z=\sqrt{16}\,\e^{\ii\frac{\pi+2k\pi}{2}}=4\,\e^{\ii\left(\frac{\pi}{2}+k\pi\right)}
=\begin{cases}4\ii,&k=0\\ -4\ii,&k=1.\end{cases}
\]
Soluzioni: $z\in\{0,4\ii,-4\ii\}$. (Disegna tre punti sull’asse immaginario.)
\end{example}

% =========================================================
\section{Serie di Fourier: costruzione e convergenza}
% =========================================================

\subsection{Setup standard su $2\pi$}
\begin{defbox}
Per $f$ $2\pi$–periodica, serie di Fourier:
\[
S_f(x)=\frac{a_0}{2}+\sum_{n=1}^\infty\left(a_n\cos nx+b_n\sin nx\right),
\]
\[
a_0=\frac{1}{\pi}\int_{-\pi}^{\pi} f(x)\,dx,\quad
a_n=\frac{1}{\pi}\int_{-\pi}^{\pi} f(x)\cos nx\,dx,\quad
b_n=\frac{1}{\pi}\int_{-\pi}^{\pi} f(x)\sin nx\,dx.
\]
\end{defbox}

\begin{ideabox}
\textbf{Scorciatoia parità}:\; $f$ pari $\Rightarrow b_n=0$;\; $f$ dispari $\Rightarrow a_0=a_n=0$.\\
Integra su intervalli comodi (qualsiasi di ampiezza $2\pi$). Per tratti costanti/lineari, spezza e usa simmetrie.
\end{ideabox}

\subsection{Teorema di Dirichlet (convergenza puntuale) e uniforme}
\begin{theorem}[Dirichlet, forma operativa]
Se $f$ è \emph{regolare a tratti} su un periodo (numero finito di discontinuità di salto, derivabile a tratti), allora:
\begin{itemize}
\item in ogni punto di \emph{continuità} di $f$, $S_f(x)\to f(x)$;
\item in ogni \emph{punto di salto} $x_0$, $S_f(x_0)\to \frac{f(x_0^-)+f(x_0^+)}{2}$.
\end{itemize}
\end{theorem}

\begin{proposition}[Convergenza uniforme: regola pratica]
Se l'estensione $2\pi$–periodica di $f$ è \textbf{continua} su $\R$, la serie $S_f$ converge \textbf{uniformemente} su $\R$. 
Se c'è \emph{anche un solo salto}, la convergenza non è uniforme sull'intero periodo.
\end{proposition}

\begin{exambox}
\textbf{Schema d’esame Fourier}:
\begin{enumerate}[(1)]
\item Disegna il \emph{periodo base} e individua parità (pari/dispari/\textit{nessuna}).
\item Scrivi esplicitamente che $f$ è \textbf{regolare a tratti} sul periodo (vale tipicamente nei compiti).
\item Calcola \emph{solo} i coefficienti non nulli (grazie alla parità).
\item Scrivi $S_f(x)$.
\item Concludi: \textbf{puntuale} via Dirichlet; \textbf{uniforme} se e solo se l’estensione periodica è continua.
\end{enumerate}
\end{exambox}

\subsection{Formule per periodo generale $2L$}
Se $f$ è $2L$–periodica:
\[
S_f(x)=\frac{a_0}{2}+\sum_{n=1}^\infty\left(a_n\cos\frac{n\pi x}{L}+b_n\sin\frac{n\pi x}{L}\right),
\quad
a_n=\frac{1}{L}\int_{-L}^{L} f(x)\cos\frac{n\pi x}{L}\,dx,
\]
\[
b_n=\frac{1}{L}\int_{-L}^{L} f(x)\sin\frac{n\pi x}{L}\,dx.
\]

\subsection{Esempi “a stile compito”}

\begin{example}[Gradino su $[0,2\pi)$]
$f(x)=\begin{cases}\dfrac{4}{\pi^2}-x,&0\le x\le \dfrac{\pi}{2},\\[0.3em]
0,&\dfrac{\pi}{2}<x<\pi,
\end{cases}$
e $2\pi$–periodica.\\
\emph{Passi:} Disegna; non è pari né dispari; $f$ regolare a tratti $\Rightarrow$ Dirichlet applicabile. 
Calcola $a_0,a_n,b_n$ spezzando gli integrali su $[0,\pi/2]$ e $(\pi/2,\pi)$, poi usa periodicità per $[-\pi,0)$. 
Scrivi la serie e concludi su puntuale e uniforme (non uniforme per via dei salti).
\end{example}

\begin{example}[Dente di sega centrato (dispari)]
$f(x)=\dfrac{\pi-x}{\pi}$ per $x\in(-\pi,\pi]$, $2\pi$–periodica.\\
\emph{Dispari} $\Rightarrow a_0=a_n=0$. Calcola solo $b_n=\frac{1}{\pi}\int_{-\pi}^{\pi}\frac{\pi-x}{\pi}\sin nx\,dx$.
Conclusione: puntuale ovunque, ai salti media dei limiti; non uniforme per via dei salti.
\end{example}

\subsection{Mini–tabella integrali rapidi}
\begin{align*}
\int \sin nx\,dx&=-\frac{\cos nx}{n}+C,&
\int \cos nx\,dx&=\frac{\sin nx}{n}+C,\\
\int x\sin nx\,dx&=\frac{\sin nx}{n^2}-\frac{x\cos nx}{n}+C,&
\int x\cos nx\,dx&=\frac{x\sin nx}{n}+\frac{\cos nx}{n^2}+C.
\end{align*}

% =========================================================
\section{Funzioni di variabile complessa: olomorfia $\Rightarrow$ analiticità}
% =========================================================

\subsection{Derivabilità complessa e olomorfia}
\begin{defbox}
$f$ è derivabile (olomorfa) in $z_0$ se esiste finito
\[
\lim_{z\to z_0}\frac{f(z)-f(z_0)}{z-z_0},
\]
\textbf{indipendentemente dal percorso} nel piano.
\end{defbox}

\subsection{Scomposizione $f(x+\ii y)=u(x,y)+\ii v(x,y)$ e Cauchy–Riemann}
\begin{theorem}[Equazioni di Cauchy–Riemann (CR)]
Se $u,v$ sono differenziabili in un punto, $f$ è olomorfa in quel punto $\iff$
\[
\pdv{u}{x}=\pdv{v}{y},\qquad \pdv{u}{y}=-\pdv{v}{x}.
\]
\end{theorem}

\begin{exambox}
\textbf{Schema d’esame Olomorfia}:
\begin{enumerate}[(1)]
\item \textbf{Dominio}: vincoli dove $f$ non è definita (denominatori $\neq0$, log rami, radici).
\item Scrivi $z=x+\ii y$ e separa $u,v$ (rationalizza se serve).
\item Afferma che $u,v$ sono differenziabili nel dominio (combinazioni di funzioni elementari).
\item Calcola $\partial u/\partial x$, $\partial u/\partial y$, $\partial v/\partial x$, $\partial v/\partial y$ e verifica CR.
\item Conclusione: insieme di olomorfia $=$ (punti del dominio dove CR valgono).\\
Frase finale: \emph{Olomorfa $\Rightarrow$ infinitamente derivabile $\Rightarrow$ analitica nello stesso insieme.}
\end{enumerate}
\end{exambox}

\subsection{Esempi “stile compito” (razionali di $z$)}
\begin{example}
$f(z)=\dfrac{1-\ii z}{1+\ii z}$. \;
Dominio: $1+\ii z\neq 0\Rightarrow z\neq \ii(-1)=-\ii$. 
Scrivi $z=x+\ii y$, calcola $u,v$ e verifica CR su $\C\setminus\{-\ii\}$. 
Conclusione: olomorfa e quindi analitica su $\C\setminus\{-\ii\}$.
\end{example}

\begin{example}
$f(z)=\dfrac{1+z}{1-z}$. \;
Dominio: $z\neq 1$. 
Stesso schema: separa $u,v$, verifica CR su $\C\setminus\{1\}$, quindi analitica lì.
\end{example}

\subsection{Osservazioni rapide}
\begin{itemize}
\item Se $f$ è polinomio in $z$, allora $f$ è olomorfa su tutto $\C$.
\item Se compaiono $\overline{z}$ in modo non banale, tipicamente \emph{non} olomorfa (CR non reggono).
\item Funzioni razionali in $z$: olomorfe su $\C$ escluse le \emph{poli} (zeri del denominatore).
\end{itemize}

% =========================================================
\section{Checklist finali e trucchetti da 30 e Lode}
% =========================================================

\subsection*{Numeri Complessi}
\begin{itemize}
\item Porta sempre in $a+\ii b$ \emph{prima} di passare a forma trig./exp.
\item \textbf{Argomento principale}: correggi il quadrante; ricordati che non è unico ($\theta+2k\pi$).
\item Radici: stesso modulo $\sqrt[n]{\rho}$; angoli $\dfrac{\theta+2k\pi}{n}$, $k=0,\dots,n-1$; \emph{disegno veloce}.
\item Equazioni coniugate: passa a $\rho,\theta$ e confronta moduli/argomenti.
\end{itemize}

\subsection*{Serie di Fourier}
\begin{itemize}
\item Subito parità: \textbf{pari} $\Rightarrow b_n=0$;\; \textbf{dispari} $\Rightarrow a_0=a_n=0$.
\item Spezza gli integrali dove la funzione cambia espressione e usa simmetrie.
\item Scrivi sempre la \textbf{frase Dirichlet} (puntuale) e la riga su \textbf{uniforme}.
\end{itemize}

\subsection*{Olomorfia}
\begin{itemize}
\item Dominio prima di tutto (evita di “dimostrare CR” fuori dominio).
\item CR sono \emph{necessarie e sufficienti} se $u,v$ sono differenziabili.
\item Conclusione testuale: “olomorfa $\Rightarrow$ analitica nello stesso insieme”.
\end{itemize}

\begin{alertbox}
\textbf{Errori da evitare}: (i) Argomento col segno/quadrante sbagliato; (ii) dimenticare $k=0,\dots,n-1$ nelle radici;
(iii) saltare la frase finale su convergenza/olomorfia; (iv) integrali Fourier senza spezzare ai punti di salto;
(v) non razionalizzare il denominatore quando serve.
\end{alertbox}

% =========================================================
\section*{Appendice A: Esempi-lampo pronti}
% =========================================================

\subsection*{Radici: $w^4=3+3\ii$}
$z=3+3\ii=\rho\,\e^{\ii\theta}$; $\rho=\sqrt{18}=3\sqrt{2}$; $\theta=\pi/4$.
\[
w_k=\sqrt[4]{3\sqrt{2}}\ \e^{\ii\frac{\pi/4+2k\pi}{4}},\quad k=0,1,2,3.
\]

\subsection*{Fourier: funzione dispari $f(x)=x$ su $(-\pi,\pi]$}
$a_0=a_n=0$;\; $b_n=\frac{1}{\pi}\int_{-\pi}^{\pi}x\sin nx\,dx=\frac{2(-1)^{n+1}}{n}$.
\[
S(x)=\sum_{n=1}^{\infty}\frac{2(-1)^{n+1}}{n}\sin nx,\quad
\text{puntuale ovunque, ai salti media; non uniforme (salti).}
\]

\subsection*{Olomorfia: $f(z)=\dfrac{z^2+1}{z-1}$}
Dominio $\C\setminus\{1\}$;\; razionale in $z$ $\Rightarrow$ olomorfa su $\C\setminus\{1\}$, quindi analitica lì.

% =========================================================
\section*{Appendice B: Mini–modulistica “frasi d’esame”}
% =========================================================

\paragraph{Dirichlet (da scrivere)}
“Si verifica che $f$ è regolare a tratti sul periodo. Per il teorema di Dirichlet, la serie di Fourier converge
puntualmente a $f$ nei punti di continuità e alla media dei limiti nei punti di discontinuità.”

\paragraph{Uniforme (continua periodica)}
“L’estensione $2\pi$–periodica di $f$ è continua su $\R$, dunque la serie converge uniformemente su $\R$.”

\paragraph{Uniforme (salti)}
“Poiché l’estensione periodica presenta punti di discontinuità, la convergenza non è uniforme sull’intero periodo.”

\paragraph{Analiticità}
“Nei punti in cui $f$ è olomorfa, per la teoria delle funzioni olomorfe, $f$ è infinitamente derivabile e analitica.”

\vfill
\begin{center}\small
\emph{Queste pagine sono pensate per riprodurre il taglio dei compiti passati: radici n–esime / Fourier / olomorfia. 
Mostra i passaggi, usa le frasi standard, e marca i disegni.}
\end{center}

\end{document}
